%&format -translate-file pdf
\documentclass{article}
\usepackage{algorithm2e}
\usepackage{amssymb}
\usepackage{listings}
\title{Homework 5}
\author{RT Hatfield}
\date{19 September 2016}
\begin{document}
    \maketitle
    \begin{enumerate}
        \item \begin{itemize}
            \item An individual \lstinline`merge` has complexity $O(l + m)$, where $l$ and $m$ are the sizes of the lists being
            merged.  We have $k$ lists of $n$ elements to merge, so the first \lstinline`merge` has complexity $O(2n)$, 
            the second has complexity $O(2n + n)$, the third has complexity $O(3n + n)$, and so forth.  We end up with 
            a complexity like $O(\sum_{i=1}^{k - 1} (i+1)n)$, which reduces to the category of $O(kn)$.
            \item Divide and conquer?  I'll divide you!  

            Really, though, I've been thinking about this and not sure how to go about it.  I keep coming up with solutions
            that involve taking pairs of the lists and merging them, but they all end up seeming to run in $O(kn)$ time.
        \end{itemize}
        \item \begin{itemize}
            \item If the majority elements of $A_1$ and $A_2$ are the same, obviously that's also the majority
            element of $A$.  If not, whichever majority element is larger is the majority element of $A$.

            The idea behind what I'm doing here is that if any element takes up over half the list, then that element
            is going to have to be repeated sequentially at some point.  I find any elements which are repeated sequentially
            and save those, ditching the rest.

            \begin{procedure}[H]
                \KwData{A list $A$}
                \KwResult{The majority element, if any, of $A$; null if none}
                \If{$A$.size == 2}{
                    \If{$A[0] == A[1]$}{
                        return $A[0]$\;
                    }
                    \Else{
                        return null\;
                    }
                }
                \Else{
                    $A_1$ = $A [ 0\dots \frac{n}{2} ]$\;
                    $A_2$ = $A [ (\frac{n}{2} + 1) \dots n ]$\;
                    $a$ = check($A_1$)\;
                    $b$ = check($A_2$)\;
                    \If{$a == b$}{
                        return $a$\;
                    }
                    \Else{
                        \If{$a == null$}{
                            return $b$\;
                        }
                        return $a$\;
                    }
                }
            \end{procedure} 
            \item This is based on the same insight I had when working out the previous algorithm, only it decides
            a little differently with less overhead.

            \begin{procedure}[H]
                \KwData{A list $A$}
                \KwResult{The majority element, if any, of $A$; null if none}
                \If{$A$.size == 2}{
                    \If{$A[0] == A[1]$}{
                        return $A[0]$\;
                    }
                    \Else{
                        return null\;
                    }
                }
                \Else{
                    \For{$i \to \frac{n}{2}$}{
                        \If{$A[i] == A[A.length - i]$}{
                            $B$.append($A[i]$)\;
                        }
                    }
                    return check($B$)\;
                }
            \end{procedure}
        \end{itemize}
    \end{enumerate}
\end{document}