%&format -translate-file pdf
\documentclass{article}
\usepackage{graphicx}
\usepackage{listings}
\usepackage{algorithm2e}
\lstset{basicstyle=\ttfamily\tiny}
\title{Project 3}
\author{RT Hatfield}
\date{19 October 2016}
\begin{document}
\maketitle

I always need the maximum subsequence + j.
I have a maximums array, where I 

First, we simply sum the whole sequence.  This is our "minimum sum."  It could be the max, but we know we don't
want anything smaller.  So we set a "max" variable to the same value.





We "crawl" across the sequence in one direction, adding one element at a time to our subsequence.  We keep track of
the sum of each one of these subsequences in some other array.  As soon as we add an element to the subsequence that
causes the sum to drop, we start scanning forward through our memorized sums until adding that sum makes the sum go 
up again.  We set the "start" pointer there and start crawling the "end" pointer again.  We repeat this process.

