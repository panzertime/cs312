%&format -translate-file pdf
\documentclass{article}
\usepackage{algorithm2e}
\title{Homework 12}
\author{RT Hatfield}
\date{17 October 2016}
\begin{document}
    \maketitle
    \begin{enumerate}
    \item \begin{itemize}
        \item (1, 1, 3, 3)
        \item \begin{itemize}
            \item I, uh, am probably not smart enough to figure this one out
        \end{itemize}
        \item
    \end{itemize}
    \item If the path from the root of the tree to the leaf corresponds to the encoding, then all we have to
    do is make sure that no character is on the path to another character, i.e. that no character is the 
    parent of any other character.  Since we have already specified that all characters are at the leaves of
    the tree, this quality is satisified.  Further, the encoding is binary, and so is the tree.  No node can
    have more than two children.  Since each node represents a partial path from the root, therefore a prefix,
    only two possible choices exist for either child: a character, which terminates the path, or another subtree,
    which continues the prefix, but by necessity does not start with the same character as the other child.
\end{enumerate}
\end{document}