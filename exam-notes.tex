%&format -translate-file pdf
\documentclass{article}
\usepackage{algorithm2e}
\title{Notes}
\author{RT Hatfield}
\date{31 August 2016}
\begin{document}
    \begin{enumerate}
        \item big-O trick: take the limit as n approaches infinity of $\frac{g(n)}{f(n)}$.  If greater than or equal to 0, then $f(n) = O(g(n)$.
        If greater than 0, less than or equal to infinity, it's big-Omega.  Between 0 and infinity, it's big-Theta.
        \item Modular exponentiation: modexp(x,y,N); if 